\nsection{Objectives}

\begin{enumerate}
    \item[\textbullet] Identify the image processing steps of a recipe such filtering, segmentation and binary processing.
    \item[\textbullet] Manage records and table.
    \item[\textbullet] Create a GA3 recipe for a given analysis problem.
    \item[\textbullet] Batch process collection of files.
\end{enumerate}

\nsection{Tips}

\begin{enumerate}
    \item[\textbullet] Open the LUTs tools (Ctrl-ALT L) before starting the General Analysis 3 interface as it blocks other menus to be launched.
    \item[\textbullet]	Workflow will modify the nd2 file content, make sure to store only the results you need and not to modify the channels.
    \item[\textbullet]	Use Alt-Up and Alt-Down to increase resp. decrease the opacity of the binaries.
    \item[\textbullet]	To add a new module to a workflow, drop it on the image, binary or measurement you want to use as input and the connection to this input will be made automatically
    \item[\textbullet]	To group similar modules in one group, drop the new module onto an existing module or select the modules to group and right click and set Group.
    \item[\textbullet]	To look for a module, use the search bar at the top.
    \textcolor{olive}{\item [\textbullet] Right click on the image and find image information for pixel to micron conversion.}
    \textcolor{olive}{\item [\textbullet] Click on the question mark on each operation (top left) for more information if needed.}
\end{enumerate}