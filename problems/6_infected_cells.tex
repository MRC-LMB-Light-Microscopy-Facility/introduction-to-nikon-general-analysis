\section{Infected cells}

\subsection*{Problem}

In this example, cells are infected by bacteria labelled with several markers. Cells are further stained with DAPI. We want to count the number of bacteria in each cell and get teh combination of positive markers to identify the type of bacteria.


\paragraph{Dataset} 
The samples have been imaged using a Nikon widefield microscope with a 20x/0.75NA objective and an Andor Neo camera (\SI{6.5}{\micro\meter}) resulting to \SI{325}{\nano\meter} pixel size in the image. The file \directory{infected cells.nd2} has the following channels:
\begin{enumerate}
    \item DAPI (blue)
    \item Bac1 (green)
    \item Bac2 (red)
    \item Bac3 (cyan)
\end{enumerate}

\paragraph{Concepts} Segmentation (Bright Spots), Data management (Aggregate Children, Append Columns).

\paragraph{Credits} Agnes Foeglein at the MRC LMB

\subsection*{Step-by-step instructions}
\begin{enumerate}
    \item Open the file ``Infected cells.nd2'' and start General Analysis 3.
    \item To segmented approximately the cells, we are going to use the DAPI channel. We can use \menu{Segmentation>Spot Detections>Bright Spots} \endnote{diameter:\SI{10}{\micro\meter}, contrast:50, Grow ticked:50} on the DAPI channel to detect spots and grow regions around them.
    \item Detect the bacteria in each of the 3 remaining channels. Here we can use again  \menu{Segmentation>Spot Detections>Bright Spots} with a smaller typical diameter\endnote{For example: typical diameter:\SI{3}{\micro\meter}, contrast: 30}.
    \item To count the number of bacteria for each type for marker per cell, we use \menu{Measurement > Object parenting > Aggregate Children} for each bacteria binary linking A to the binary obtained from the DAPI channel.
    \item We can use \menu{Data Management>Basic>Append Columns} to join the 3 tables in one as the index of the cells are matching.
\end{enumerate}

% Use bac2 as bacteria marker and check for the other labels to encode the bacteria identify.