\subsection{Particle tracking}

\paragraph{Problem}
Here to illustrate particle tracking, moving beads have been simulated using a theoritical point spread function.

\paragraph{Dataset} The file \directory{5 particle tracking/synthetic beads.nd2} has a single channel and 20 time points.

\paragraph{Credits} Dina Ratsimandresy at the MRC-LMB

\paragraph{Objectives} Segmentation (Bright Spots), Tracking (Time \& centroid, Track Particles, Accumulate \& group, Speed), Measurement (Mean Obj Intensity), Results (Linechart)


\paragraph{Step-by-step instructions}

\begin{enumerate}
    \item Open the file \directory{5 particle tracking/synthetic beads.nd2} and open \menu{Image>General Analysis 3}.
    \item To detect the particles, use \menu{Segmentation>Spot Detections>Bright Spots}
    \endnote{typical diameter:0.5 and contrast 2}. 
    \item Use the \menu{Tracking>2D Object Position>Time \& Centroid} to extract the centroid of each particle and store it in a table.
    \item Measure the mean intensity of each particle using \menu{Measurement>Object intensity>Mean Obj Intensity} and by adding it to the previous measurement.
    \item Track the particles using \menu{Tracking > 2D Tracking > Track Particles}. Check that the columns matches the ones from the centroid table. We can see that a new column ``TrackId'' is added, but only one frame have been processed
    \endnote{stdev multiplicative factor:4}.
    \item To tracks across the frames of the whole image sequence, we need to add \menu{Tracking > Tracks > Accumulate Tracks}, now the table has grouped the objects by tracks and a new tracking icon appeared at the top of the table.  
    \item We can display the tracks using the tracking icon at the top of the table.
\end{enumerate}


