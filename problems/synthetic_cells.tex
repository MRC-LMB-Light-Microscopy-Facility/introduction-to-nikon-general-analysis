\section{Synthetic cells}

\subsection{Problem}
In this problem, we make abstraction of the preprocessing steps and focus on the manipulation of binary layers. The image is a sketch representation of three cells with nuclei, cyctosol and some putative vesicles. We want to count the number of vesicle and the area of the nuclei area for each cell.

\begin{center}
\begin{tabular}{rrr}
    Cell ID& Number of endosomes& Nuclei area[px$^2$]\\
    1&8&6562\\
    2&13&7668\\
    3&9&11895\\
\end{tabular}   
\end{center}

\paragraph{Dataset:} The image is \path{synthetic_cells.nd2} with the following channels:
\begin{itemize}\setlength\itemsep{0em}
    \item red: endosomes
    \item green: cytoplasm
    \item blue: nuclei
\end{itemize}

\paragraph{Concepts:} object parenting (ParentID, Child ID, Aggregate Children), data management (GroupRecords, AggegateRows, ModifyColumns, JoinRecords, Aggregate Children), measurement (ObjectArea)

% \begin{description}
%     \item[Task:] For each cell, count the number of red dots and measure the area of the nuclei.
%     \item[Data:] ``synthetic cells.nd2'' with three channels: dots (red), cytoplasm (green) and nuclei (blue). 
%     \item[Topics:] object parenting (ParentID, Child ID, Aggregate Children), data management (GroupRecords, AggegateRows, ModifyColumns, JoinRecords, Aggregate Children), measurement (ObjectArea)
%     \item[Duration:] 30 min
%     \item[Step by step:]
% \end{description}

\subsection{Step-by-step instructions}

\begin{enumerate}
    \item Load the image ``synthetic cells.nd2''.

    \item Create binary masks for all the channels with the names ``endosomes'', ``cells'' and ``nuclei'' \endnote{Use a threshold with value 128-255 for example.}. 

    \item For creating a parent-child hierarchy associating the endosomes to the cells and count the number of endosomes per cell, add the action \menu{Measurement>Object parenting>Aggregate Children} with A (parent): ``cells'' binary and B (child): ``endosomes''. 

    % \begin{description}
    %     \item[Option 1] Associate a parent to each endosome and group the records
    %     \begin{itemize}
    %         \item To associate a cell to each dot,  use \menu{Measurement > Object parenting > Parent ID} linking A (parent) to the ``cytoplasm'' binary and B (child) to the ``dots'' binary.
    %         \item Group the row by adding a module \menu{Data Management > Grouping > Group Records} and use cytoplasmID.    
    %         \item Add the module \menu{Data Management > Grouping >  Aggregate Rows} and set the line ``ObjectId'' to ``Count'' in AggregateRows. 
    %     \end{itemize}

    %     \item[Option 2] Associate a list of child to each cytoplasm and group the records
    %     \begin{itemize}
    %     \item Associate for each cytoplasm, the list of endosomes using \menu{Measurement > Object parenting > Child ID} linking A (parent) to the ``cytoplasm'' binary and B (child) to the ``endosomes'' binary.
    %     \item Group the row by ObjectId using the action \menu{Data Management > Grouping > Group Records} 
    %     \item Add the action \menu{Data Management > Grouping > Aggregate Rows} and set ``dotsID'' to ``Count'' 
    %     \end{itemize}

    %     \item[Option 3] The element ``Aggregate children'' enable to compute parenting and aggregate by parent at once.
    %     \begin{itemize}
    %         \item Drop the module \menu{Measurement>Object parenting>Aggregate Children} on the ``cytoplasm'' binary and link B (child) to the ``endosomes'' binary. 
    %     \end{itemize}
    % \end{description}

    \item Measure the area of each nuclei using \menu{Measurement > Object Size > Object Area}.
    
    \item Add the cell ID to the records of nuclei area using \menu{Measurement > Object parenting > Parent Id} with A (parent) to the ``cell'' and B (child) to the ``nuclei'' binary. 
    
    \item Use \menu{Data Management > Basic > Join Records} to join using ``Object Id'' for tables A and B linking the area measurements and the parenting.
    
    \item Group the row by ``CellId'' using the action \menu{Data Management > Grouping > Group Records} 
    
    \item Add the action \menu{Data Management > Grouping > Aggregate Rows} and set ``Object Area'' to ``Sum'' 
    
    \item Use \menu{Data Management > Basic > Join Records} to join using ``Cell Id'' for tables A and B.
    
    \item Use \menu{Data Management > Basic > Rename Columns} to keep only the necessary columns.

\end{enumerate}
