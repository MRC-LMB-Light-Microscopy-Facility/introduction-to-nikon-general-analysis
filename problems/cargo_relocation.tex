\section{Selective cargo tethering}

\subsection{Problem}

In this example, we are interested into quantifying the capture of syntaxin-16 cargos vesicles by the mitochondria when the protein TBC1D23 is relocated to the mitochondria.  Prepared cells were imaged using a Zeiss confocal microscope using a 63x/1.4NA oil immersion lens.

\paragraph{Dataset} We will use the file ``Zeiss1344.lsm'' which have the following channels:
\begin{itemize}\setlength\itemsep{0em}
    \item Ch2-T1 : Golgi maker GM130
    \item ChS2-T2 : cargo protein
    \item ChS1-T3 : mitochondria
    \item Ch1-T4 : nuclei marked with DAPI
\end{itemize}

\paragraph{Concepts} Segmentation with seeded watershed (Threshold, DistanceFunction, Watershed, GrowObjects), colocalization (Pearson Coeff), data management (AppendColumns, Binning, GroupRecords, AppendColumns), display (Barchart).

\paragraph{Credit}  Alison Gillingham from Sean Munro's group at the MRC-LMB

\paragraph{Reference} Jérôme Cattin-Ortolá et al., Cargo selective vesicle tethering: The structural basis for binding of specific cargo proteins by the Golgi tether component TBC1D23. Sci. adv.10,eadl0608(2024). 

DOI:10.1126/sciadv.adl0608

\subsection{Step-by-step instructions}
\begin{enumerate}
    \item Open the image ``Zeiss1344.lsm''.
    \item Segment the cells using a seeded watershed approach by following the three steps below:
    \begin{enumerate}
        \item Create a mask for the cells using the Golgi marker (Ch2-T1) channel. 
median filter 2px
threshold 8-255
smooth, clear, 


\soln Use \menu{Preprocessing > Convolution > Gaussian Filters} with sigma 2 and \menu{Segmentation>Threshold} with range [1,255]. \solnend
        \item Segment the nuclei. \soln Use \menu{Segmentation>Threshold} with range [30,255], Smooth 5x, Clean OFF, FillHoles OFF, Separate OFF and set the minimal size of the object to \SI{5}{\micro\meter}. \solnend
        \item Separate the touching cells using the nuclei. Drop \menu{Binary processing > Detect > Distance function} onto the Golgi marker (Ch2-T1) binary. Use the \menu{Binary processing> Region growing > Watershed} select the type ``From Bright Regions''. Use \menu{Binary operations > And} linking the result of the watershed (Ch1-T4) and the initial cell mask (Ch2-T1) to restrict the mask to the cells.
    \end{enumerate}
    \item Measure the colocalization of the cargo protein with the Golgi marker (resp the mitochondria). Use \menu{Measurement > Object ratiometry > Pearson Coeff} to measure the colocalization coefficient between the Golgi (Ch2-T1, link to B) and cargo protein (ChS2-T2, link to C) within the segmented region (binary to link to A). Use \menu{Measurement > Object ratiometry > Pearson Coeff} to measure the colocalization coefficient between the mitochondria (ChS1-T3, link to B) and cargo protein (ChS2-T2, link to C) within the segmented region (binary to link to A).
    \item Measure the mean intensity of the mitochondria channel.
    \item Merge the three records table together. \soln Use \menu{Data Management > Basic > AppendColumns}. \solnend
    \item At this point make sure that the workflow doesn’t remove any channel.
    \item Batch process all the other lsm files. Close the GA3 interface and navigate to \menu{Image>Batch GA3}.
    \item Save all the .lsm files as .nd2 files into a different folder while tick the "keep original" option.
    \item Create a figure with the two conditions based on low and high level of mitochondria expression.
\end{enumerate}
