\pagebreak
\nsection{Cargo relocation}

\begin{description}
    \item[\entry{Task:}] Measure the colocalization between a cargo protein and a Golgi marker and the cargo protein and the mitochondria.
    \item[\entry{Data:}] ``Zeiss1344.lsm'' with four channels Ch2-T1 (Golgi maker GM130), ChS2-T2 (cargo protein), ChS1-T3 (mitochondria), Ch1-T4 (nuclei marked with DAPI).
    \item[\entry{Topics:}] Segmentation with seeded watershed (Threshold, DistanceFunction, Watershed, GrowObjects), colocalization (Pearson Coeff), data management (AppendColumns, Binning, GroupRecords, AppendColumns), display (Barchart).
    \item[\entry{Reference:}] 
        Jérôme Cattin-Ortolá et al., Cargo selective vesicle tethering: The structural basis for binding of specific cargo proteins by the Golgi tether component TBC1D23.Sci. adv.10,eadl0608(2024).
        DOI:10.1126/sciadv.adl0608
    \item[\entry{Acknowledgement:}] Alison Gillingham
    \item[\entry{Duration:}] 40 min
    \item[\entry{Step by step:}]
\end{description}

\begin{enumerate}
    \item Open the image ``Zeiss1344.lsm''.
    \item Segment the cells using a seeded watershed approach by following the three steps below:
    \begin{enumerate}
        \item[2.1.] Create a mask for the cells using the Golgi marker (Ch2-T1) channel. \soln Use \menu{Preprocessing > Convolution > Gaussian Filters} with sigma 2 and \menu{Segmentation>Threshold} with range [1,255]. \solnend
        \item[2.2.] Segment the nuclei. \soln Use \menu{Segmentation>Threshold} with range [30,255], Smooth 5x, Clean OFF, FillHoles OFF, Separate OFF and set the minimal size of the object to \SI{5}{\micro\meter}. \solnend
        \item[2.3.] Separate the touching cells using the nuclei. Drop \menu{Binary processing > Detect > Distance function} onto the Golgi marker (Ch2-T1) binary. Use the \menu{Binary processing> Region growing > Watershed} select the type “From Bright Regions”. Use \menu{Binary operations > And} linking the result of the watershed (Ch1-T4) and the initial cell mask (Ch2-T1) to restrict the mask to the cells.
    \end{enumerate}
    \item Measure the colocalization of the cargo protein with the Golgi marker (resp the mitochondria). Use \menu{Measurement > Object ratiometry > Pearson Coeff} to measure the colocalization coefficient between the Golgi (Ch2-T1, link to B) and cargo protein (ChS2-T2, link to C) within the segmented region (binary to link to A). Use \menu{Measurement > Object ratiometry > Pearson Coeff} to measure the colocalization coefficient between the mitochondria (ChS1-T3, link to B) and cargo protein (ChS2-T2, link to C) within the segmented region (binary to link to A).
    \item Measure the mean intensity of the mitochondria channel.
    \item Merge the three records table together. \soln Use \menu{Data Management > Basic > AppendColumns}. \solnend
    \item At this point make sure that the workflow doesn’t remove any channel.
    \item Batch process all the other lsm files. Close the GA3 interface and navigate to \menu{Image>Batch GA3}.
    \textcolor{olive}{\item Save all the .lsm files as .nd2 files into a different folder while tick the "keep original" option.}
\end{enumerate}

\begin{description}
    \item[\entry{Optional:}] make a figure with the two conditions based on low and high level of mitochondria expression.
\end{description}