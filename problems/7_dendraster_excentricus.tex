\subsection{\textit{Dendraster excentricus} embryos} \label{sec:dendraster-excentricus}

\paragraph{Problem}
A developing sand dollar embryo expressing GFP-histone H2B have been images using video-microscopy every \SI{30}{\second} for \SI{100}{\hour}. We want to segment and visualize the number of cells at each frame in a first step and later track the cells over time.

\paragraph{Dataset} The file \directory{7 dendraster excentricus/cil-15798.tif} has one channel and 200 time points.

\paragraph{Objectives} 
preprocessing (Rolling Ball), 
segmentation (Threshold), 
object tracking, 
measurement (Object Count), 
results (Linechart)

\paragraph{Reference} 
George Von Dassow (2011) CIL:15798, Dendraster excentricus. CIL.Dataset. 

https://doi.org/doi:10.7295/W9CIL15798


% \begin{description}
%     \item[Task:] Track the cells along the division and count the number of cells over time.
%     \item[Data:] ``cell division.tif'' with one channel and 200 time points.
%     \item[Topics:] preprocessing (Rolling Ball), segmentation (Threshold), object tracking, measurement (Object Count), results (Linechart)
%     \item[Reference:] 
%         George Von Dassow (2011) CIL:15798, Dendraster excentricus. \\
%         CIL.Dataset. https://doi.org/doi:10.7295/W9CIL15798
%     \item[Duration:] 40 min
%     \item[Step by step:]
% \end{description}

\paragraph{Step-by-step instructions}

\begin{enumerate}
    \item Open the image ``cell division.tif'' and open \menu{Image > new GA3 recipe}.
    \item Segment the nuclei using for example \menu{Preprocessing>Background>Rolling Ball}\endnote{radius:10px} and \menu{Segmentation>Threshold>Threshold}\endnote{range: [10, INF], smooth:8x, clean:1x, fill holes:off, separate:1x and size:5-INF.}
    \item To count the number of nuclei in the field of view we can use  \menu{Measurement>Whole Field>Object Count}.
    \item However, this gives only the count only for a single frame, to get the results for each frame, we can use \menu{Data Management > Accumulate Records}.
    \item Next, let's visualize the number of nuclei vs time using \menu{Results>Graphs>Linechart}.
    \item We could also track the individual nuclei droping \menu{Tracking>Track Objects} and \menu{Tracking>Tracks>Accumulate Tracks}.
\end{enumerate}
