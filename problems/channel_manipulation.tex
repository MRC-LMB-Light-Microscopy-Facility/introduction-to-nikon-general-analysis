\pagebreak
\nsection{Channel manipulation}

\begin{description}
    \item[\entry{Task:}] Remove a channel from a dataset.
    \item[\entry{Data:}] The file ``isim\_001.nd2'' has two channels Red and Blank.
    \item[\entry{Topics:}] Store results, Batch GA3.
    \item[\entry{Duration:}] 10 min.
    \item[\entry{Step by step:}]
\end{description}

\begin{enumerate}
    \item Open the image “isim\_001.nd2” using “File > Open” or a drag and drop.
    \item Open the General analysis 3 interface using “Image > New GA3 recipe” or a right click on the background, select “Analysis Controls > Analysis Explorer”, select “Create New > General Analysis 3”. 
    \item Select the Blank channel and click “Do not Store this Result”.
    \item Save the recipe, using you name as a prefix. \textcolor{olive}{"Save" saves the recipe in the software; "Export" allows you to save it locally with a specified path on the PC. When using an existing recipe, it can be imported by right clicking on the background, select “Analysis Controls > Analysis Explorer”.}
    \item Navigate to “Image > Batch GA3” or right click on the background, select “Batch GA3”.
    \item Press the first icon “Add files”. Select the recipe and the three isim\_xxx.nd2 files.
    \item Tick “Keep Original”.
    \item Press “Run”.
    \item Open the processed images.
\end{enumerate}