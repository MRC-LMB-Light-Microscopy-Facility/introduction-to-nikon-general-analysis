\section{Particle tracking}

\subsection{Problem}

\paragraph{Dataset} ``synthetic beads.nd2'' with one channel and 20 time points.

\paragraph{Concepts} Segmentation (Bright Spots), Tracking (Time \& centroid, Track Particles, Accumulate \& group, Speed), Measurement (Mean Obj Intensity), Results (Linechart)

\subsection{Step-by-step instructions}

\begin{enumerate}
    \item Open the file ``synthetic beads.nd2'' and create a new GA 3 workflow.
    \item Detect the particles in the image. Use \menu{Segmentation>Spot Detections>Bright Spots} with diameter 0.5 and contrast 2. 
    \item Use the \menu{Tracking>2D Object Position>Time \& Centroid} to extract the centroid of each particle.
    \item Measure the mean intensity of each particle using \menu{Measurement>Object intensity>Mean Obj Intensity}.
    \item Merge the two tables to have the centroid and the mean intensity of each particle in one table. 
    
    Use \menu{Data Management>Basic>Join Records} using the ObjectId to link the tables A and B. \solnend
    \item Track the particles over time. Use \menu{Tracking > 2D Tracking > Track Particles} with ``Time Column'' set to "TimeLapseIndex" or "Time"\soln and ``Maximum Speed'' set to 5 \solnend. Add \menu{Tracking > Tracks > Accumulate \& Group}. Tracking objects applies to the image while track particle applies to the records; max speed corresponds to max speed of the objects between consecutive frames (however not working in the software at the moment), while max gap size is max allowed skipped frames.
    \item Measure the speed of each tracked particle. \soln Use \menu{Tracking > Tracking features > Speed} and set ``DiffTime'' to Time. \solnend
    \item Display the tracks. Use \menu{Results > Linechart}, in the Data tab use ``CentroidX'' for theh ``X Axis'' and ``CentroidY'' for the ``Y Axis Left''.
\end{enumerate}

\begin{description}
    \item[Optional:] Use the tracking module on the binaries generate by the workflow.
\end{description}
